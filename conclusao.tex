%!TEX root = index.tex
\chapter{Conclusão}
\label{cha:trabalhos_futuros}

O trabalho em questão permitiu analisar e ilustrar a percepção de cada um dos \textit{stakeholders} em relação ao laboratório, podendo ser utilizado como consulta para próximos projetos a serem desenvolvidos no laboratório. Deve-se lembrar sempre do contexto no qual o laboratório foi inserido no Departamento de Engenharia de Produção, como parte do ecossistema de inovação e empreendedorismo incentivado pelo departamento, com foco em ensino e extensão. O laboratório Ocean representa o funcionamento de uma parceria Universidade-Empresa que foge das parcerias mais tradicionais - normalmente voltadas à frente de pesquisa - para trazer um \textit{know-how} técnico de classe mundial a uma das maiores Universidades do mundo, que também é uma fábrica de empreendedores no país.

Embora seja uma parceria de cogestão entre a Samsung e o PRO, as análises foram feitas acima de ambas as gestões, analisando as principais interações e operações do laboratório Ocean e unificando as deficiências e oportunidades de ambas as partes em um único lugar. Como resultado deste trabalho, o PRO agora tem acesso aos pontos levantados pelos cursistas e a Samsung tem acesso à percepção dos alunos diante do laboratório, oferecendo um novo ponto de vista da gestão para cada um. O autor acredita que se a principal cogestão efetiva entre partes for apenas a gestão do espaço físico do laboratório, a parceria tem escopo limitado e não usufrui de todo seu potencial, portanto é necessário ser feito um alinhamento estratégico e alinhar os problemas e necessidades de tempos em tempos, pois serão nas interações ou nos conflitos gerados nesses momentos que a parceria universidade-empresa poderá mostrar seu verdadeiro valor.

\section{Trabalhos futuros} % (fold)
\label{sec:trabalhos_futuros}

Devido à natureza holística do trabalho como uma análise geral do laboratório, alguns pontos poderiam ser explorados com maior profundidade, como a análise de questionários dos cursistas básicos e as propostas para o pontos estratégicos levantados com a análise de \textit{stakeholders}. Esses pontos, se explorados com metodologia apropriada e aprofundada, poderiam gerar projetos interessantes para a Samsung e o PRO, ou até para os outros \textit{stakeholders} envolvidos.