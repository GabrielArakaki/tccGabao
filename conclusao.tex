%!TEX root = index.tex
\chapter{Conclusão}
\label{cha:trabalhos_futuros}

O trabalho em questão permitiu analisar e ilustrar a percepção de cada um dos \textit{stakeholders} em relação ao laboratório, podendo ser utilizado como consulta para próximos projetos a serem desenvolvidos no laboratório. Deve-se lembrar sempre do contexto no qual o laboratório foi inserido no Departamento de Engenharia de Produção, como parte do ecossistema de inovação e empreendedorismo incentivado pelo departamento, com foco em ensino e extensão. O laboratório Ocean representa o funcionamento de uma parceria Universidade-Empresa que foge das parcerias mais tradicionais - normalmente voltadas à frente de pesquisa - para trazer um \textit{know-how} técnico de classe mundial a uma das maiores Universidades do mundo, que também é uma fábrica de empreendedores no país.

\section{Discussão} % (fold)
\label{sec:discuss_o}

A partir das análises realizadas, foi possível verificar um certo desalinhamento da percepção que os alunos têm do laboratório ao comparar com os outros \textit{stakeholders}, pois a sua visão é muito voltada ao laboratório como estrutura física, como alternativa à biblioteca e ao laboratório de informática. Uma maior divulgação entre os alunos dos programas do laboratório e uma reforma da biblioteca poderiam contribuir com uma mudança de percepção dos alunos.

\section{Trabalhos futuros} % (fold)
\label{sec:trabalhos_futuros}

Devido à natureza holística do trabalho como uma análise geral do laboratório, alguns pontos poderiam ser explorados com maior profundidade, como a análise de questionários dos cursistas básicos e as propostas para o pontos estratégicos levantados com a análise de \textit{stakeholders}. Esses pontos, se explorados com metodologia apropriada e aprofundada, poderiam gerar projetos interessantes para a Samsung e o PRO, ou até para os outros \textit{stakeholders} envolvidos.