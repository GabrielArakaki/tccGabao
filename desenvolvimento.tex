%!TEX root = index.tex
\chapter{Desenvolvimento}
\label{cha:desenvolvimento}

\section{Identificação dos Stakeholders}
\label{sec:identificacao_stakeholders}

Ao longo da implementação do Ocean, foram definidos os seguintes stakeholders:

\begin{itemize}
\item PRO ( Chefia e COC [Zancul e Mesquita] e Pesquisa [Roberta])
\item Samsung (Selber, Conejo)
\item Alunos (CAEP, RD)
\item Usuários Externos ( Reinaldo [IGDA], Sakuda [Abragames], IoT)
\end{itemize}

\subsection{PRO}
\label{sec:con_pro}

Segundo \citeonline{jeeModels}, o ensino de engenharia pode ser dividido em 3 principais modelos: Acadêmico, Market-Driven e Integrativo.

\begin{table}[h]
\begin{center}
\caption{Modelos de ensino de engenharia}
\label{tab:modelos_ensino_tab}
{\def\arraystretch{2}\tabcolsep=10pt
\begin{tabular}{>{\raggedright}p{0.2\linewidth}>{\raggedright\arraybackslash}p{0.2\linewidth}>{\raggedright\arraybackslash}p{0.2\linewidth}>{\raggedright\arraybackslash}p{0.2\linewidth}}
\hline
     & Modelo Acadêmico & Modelo \textit{Market-Driven} & Modelo Integrativo \\ \hline
     Percepção de Engenharia & Ciência Aplicada & Inovação Tecnológica & Serviço Público \\
     Papel Social & Consultor, Especialista & Empreendedor, Gestor & Cidadão, Agente de Mudanças \\
     Perspectiva Institucional & Universidade Científica & Universidade Empreendedora & Universidade Ecológica  \\
	 Exemplos de Disciplinas & Cálculo, Estatística & Empreendedorismo, Desenvolvimento de Produto & Sustentabilidade, Problemas da Sociedade \\ \hline
\end{tabular}%
}
\caption* {Fonte: Adaptado de \citeonline{jeeModels}}
\end{center}
\end{table}

Dentro desse contexto, é possível observar uma grande sinergia entre o laboratório e o modelo \textit{Market-Driven}, pois o departamento pode o utilizar para auxiliar no desenvolvimento de engenheiros para estarem alinhados com as necessidades do mercado, sendo este representado por uma das empresas com maior tecnologia de ponto a nivel global.

\subsection{Samsung}
\label{sec:con_samsung}

Ainda, por estar dentro da universidade, a Samsung têm acesso direto a especialistas com \textit{know how} restrito a poucas instituições no mundo e acesso a graduandos e pós-graduandos com potencial para futuros recrutamentos 

\subsection{Alunos}
\label{sec:con_alunos}

A presença de um laboratório como este também ajuda a fomentar a cultura de empreendedorismo dentro da universidade, pois deixa os alunos próximos ao desenvolvimento de software, uma das principais bases de criação de novas \textit{startups}, devido ao baixo custo de aprendizado e investimento e alto valor gerado no curto e médio prazo. Ainda, segundo \citeonline{entrepreneurship}, os estudantes de engenharia experienciam o empreendedorismo de 4 maneiras: 

\begin{enumerate}
\item Primeiro passo para o auto-aprendizado
\item Preparação para a vida no trabalho
\item Caminho para ser autônomo
\item Desenvolvimento de liderança e responsabilidade de um time
\end{enumerate}

\subsection{Usuários Externos}
\label{sec:con_usuarios}

Segundo a 27\textsuperscript{a} Pesquisa de Anual do uso de TI, realizada pela Fundação Getúlio Vargas (FGV), o número de smartphones em uso no Brasil gira atualmente em torno de 168 milhões de dispositivos. \cite{tifgv} Não obstante, além do alto número de smartphones, o Brasil também se mostra presente no mercado de outros dispositivos inteligentes, com previsão de movimentação de US\$4,1 bilhões no Brasil com IOT, segundo a assessoria de imprensa da IDC Brasil. \cite{idc}

É nesse cenário de alto crescimento do uso de novas tecnologias no Brasil que o mesmo se mostra como um grande mercado para produtos inerentes ao uso de dispositivos inteligentes, como aplicativos e games. Dentro desse contexto, jovens interessados pelo desenvolvimento desse mercado no país podem utilizar o Ocean para realizar diferentes cursos nessas áreas, desde aulas para iniciantes até cursos mais avançados.

Além do uso aplicado diretamente nessa área de dispositivos portáteis, a programação desenvolvida nessas atividades pode ser extendida para outras áreas de desenvolvimento, tornando os jovens mais capacitados para qualquer área tecnológica. Segundo a ONG Code.org, financiada por fundadores das maiores empresas de tecnologia do mundo como Mark Zuckerberg e Bill Gates, o número de empregos para programadores cresce exponencialmente, ao passo que o ensino de programação nas escolas não acompanha o mesmo ritmo, o que gerará uma falta de profissionais de TI em um futuro próximo. Juntamente a essa informação, o departamento de estatísticas de trabalho dos Estados Unidos (\textit{Bureau of Labor Statistics}) estima que o número de empregos para programadores dentro dos EUA diminuirá em até 8\%, pois mais profissionais deverão ser recrutados fora do país, devido ao baixo custo e a flexibilidade de trabalho remoto permitida pela programação. \cite{bls}

\section{Levantamento das necessidades} % (fold)
\label{sec:necessidades}

\section{Definição de processos} % (fold)
\label{sec:processos}

\section{Definição de indicadores} % (fold)
\label{sec:indicadores}