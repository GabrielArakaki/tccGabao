%!TEX root = index.tex
\chapter{Desenvolvimento}
\label{cha:desenvolvimento}

\section{Benchmark: Parcerias empresa-universidade} % (fold)
\label{sec:cases}

\subsection{PUC-RS}

Os parques tecnológicos são um movimento... (CITAÇAO DE TEXTO SOBRE PARQUES TECNOLOGICOS)

No Brasil há varios parques tecnológicos de destaque, como o Porto Digital em Recife, o Parque Tecnológico de São José dos Campos, o Parque Tecnológico da UFRJ e a TECNOPUC. Dentre esse parques, a TECNOPUC se mostra um excelente caso de sucesso entre universidade e empresa, realizando um trabalho muito ativo.

O portfólio de empresas que tiveram algum tipo de interação com o TECNOPUC encobre grandes empresas do segmento de tecnologia, como a própria Samsung, Microsoft, Motorola e Dell quanto empresas de menor tamanho e até instituições financeiras, como o HSBC.

O Parque Científico e Tecnológico da PUCRS (Tecnopuc) estimula a pesquisa e a inovação por meio de uma ação simultânea entre academia, instituições privadas e governo. Empresas de diferentes portes, entidades e centros de pesquisa da própria Instituição estão sediados nos dois sites: em Porto Alegre e em Viamão, ambos no Estado do Rio Grande do Sul - Brasil. Atualmente, o Tecnopuc abriga 120 organizações, somando mais de 6,3 mil postos de trabalho. Em Porto Alegre, capital do Rio Grande do Sul, a área é de 11,5 hectares e mais de 50 mil metros² de área construída.

O TECNOPUC é um Parque Científico e Tecnológico multissetorial, focado em quatro áreas:
Tecnologia da Informação e Comunicação;
Energia e Meio Ambiente;
Ciências da Vida;
Indústria Criativa.
Estas áreas temáticas foram definidas a partir da competência acadêmica da Universidade, envolvendo grupos de pesquisa científica e tecnológica e cursos de pós-graduação (mestrado e doutorado), associada à existência de demanda da sociedade. 

O Tecnopuc integra a INOVAPUCRS - Rede de Inovação e Empreendedorismo da PUCRS. Dela, também fazem parte:
- Núcleo Empreendedor, que apoia e incentiva ações inovadoras e empreendedoras;
- Ideia - Instituto de Pesquisa e Desenvolvimento, que estimula o desenvolvimento de projetos de pesquisa científica e tecnológica e oferece infraestrutura laboratorial, espaço físico e prototipagem;
- Incubadora Raiar, que abriga startups, spin-offs, além de projetos pré-incubados, incentivando o empreendedorismo e preparando empresas para o mercado;
- Centro de Inovação, uma parceria com a Microsoft, que objetiva acelerar o uso de novas tecnologias e desenvolver programas de qualificação;
- Labelo - Laboratórios Especializados em Eletroeletrônica, que apoia o fortalecimento e a qualificação dos produtos para atender a regulamentos e normas internacionais por meio de ensaios de avaliação de conformidade de produtos e emitindo relatórios de avaliação metrológica e certificados de calibração;
- AGT - Agência de Gestão Tecnológica, que viabiliza a realização de projetos de pesquisa;
- ETT - Escritório de Transferência de Tecnologia, que avalia a invenção e garante a propriedade intelectual, preservando direitos e transferindo resultados aos pesquisadores;
- AGE - Agência de Gestão de Empreendimentos, que atua com o objetivo de estruturar e desenvolver estratégias de fundraising, novos empreendimentos e serviços especializados com base no conhecimento e tecnologias geradas na Universidade;
- NAGI - Núcleo de Apoio à Gestão da Inovação (NAGI), que elabora diagnósticos para a identificação do estágio em que as organizações se encontram em relação à inovação, oferecendo assessoria e capacitação.
Juntas, as unidades atuam estimulando o processo de inovação e empreendedorismo da PUCRS.

(INSTITUCIONAL)

Objetivos: 

Atrair empresas de pesquisa e desenvolvimento (P,D&I) para trabalhar em parceria com a Universidade;
Promover a criação e o desenvolvimento de novas empresas de base tecnológica;
Atrair projetos de pesquisa e desenvolvimento tecnológico em geral;
Estimular a inovação e a interação empresas-Universidade;
Gerar uma sinergia positiva entre o meio acadêmico e o empresarial;
Atuar de forma coordenada com as esferas governamentais, particularmente no âmbito do Projeto Porto Alegre Tecnópole.

Missão:

Criar uma comunidade de pesquisa e inovação transdisciplinar por meio da colaboração entre academia, empresas e governo visando aumentar a competitividade dos seus atores e melhorar a qualidade de vida de suas comunidades.

Visão:

Em 2015 o TECNOPUC será referência nacional e internacional pela relevância das pesquisas com a marca da inovação, promovendo o desenvolvimento técnico, econômico e social da região.



A TECNOPUC tomou uma série de iniciativas para levar o parque tecnológico até onde está hoje. Segundo \citeonline{tecnopuc}, o TECNOPUC é estratégico para o processo de interação entre a universidade e as empresas, sendo o grande facilitador da comunicação entre ambas as partes.

Segundo 

\section{Identificação dos Stakeholders}
\label{sec:identificacao_stakeholders}

Ao longo da implementação do Ocean, foram definidos os seguintes stakeholders:

\begin{itemize}
\item PRO ( Chefia e COC [Zancul e Mesquita] e Pesquisa [Roberta])
\item Samsung (Selber, Conejo)
\item Alunos (CAEP, RD)
\item Usuários Externos ( Reinaldo [IGDA], Sakuda [Abragames], IoT)
\end{itemize}

\subsection{PRO}
\label{sec:con_pro}

Segundo \citeonline{jeeModels}, o ensino de engenharia pode ser dividido em 3 principais modelos: Acadêmico, Market-Driven e Integrativo.

\begin{table}[h]
\begin{center}
\caption{Modelos de ensino de engenharia}
\label{tab:modelos_ensino_tab}
{\def\arraystretch{2}\tabcolsep=10pt
\begin{tabular}{>{\raggedright}p{0.2\linewidth}>{\raggedright\arraybackslash}p{0.2\linewidth}>{\raggedright\arraybackslash}p{0.2\linewidth}>{\raggedright\arraybackslash}p{0.2\linewidth}}
\hline
     & Modelo Acadêmico & Modelo \textit{Market-Driven} & Modelo Integrativo \\ \hline
     Percepção de Engenharia & Ciência Aplicada & Inovação Tecnológica & Serviço Público \\
     Papel Social & Consultor, Especialista & Empreendedor, Gestor & Cidadão, Agente de Mudanças \\
     Perspectiva Institucional & Universidade Científica & Universidade Empreendedora & Universidade Ecológica  \\
	 Exemplos de Disciplinas & Cálculo, Estatística & Empreendedorismo, Desenvolvimento de Produto & Sustentabilidade, Problemas da Sociedade \\ \hline
\end{tabular}%
}
\caption* {Fonte: Adaptado de \citeonline{jeeModels}}
\end{center}
\end{table}

Dentro desse contexto, é possível observar uma grande sinergia entre o laboratório e o modelo \textit{Market-Driven}, pois o departamento pode o utilizar para auxiliar no desenvolvimento de engenheiros para estarem alinhados com as necessidades do mercado, sendo este representado por uma das empresas com maior tecnologia de ponto a nivel global.

\subsection{Samsung}
\label{sec:con_samsung}

De forma a incentivar a cultura, o esporte, o social e o desenvolvimento do país, foram criadas várias leis de incentivo para empresas a investirem nessas frentes a troca de uma renúncia fiscal.  Normalmente, o governo abre mão de parte dos impostos da empresa pois os mesmos serão destinados a outros projetos de benefício da sociedade. Do lado da empresa é extremamente positivo, pois esse incentivo pode ser usado tanto para reforçar a imagem da empresa quanto para gerar um retorno financeiro, fatos que não ocorreriam caso o mesmo investimento fosse aplicado em forma de impostos.

Entre essas leis encontra-se a Lei 8.248/91, conhecida como lei da informática, que foi sancionada em Outubro de 1991 pelo então presidente Fernando Collor. Dentro desta Lei o principal benefício é a redução da alíquota do IPI de 15\% para 3\% até 2029. Em contrapartida, a empresa beneficiada por essa Lei se compromete a investir até 4\% do faturamento de determinado segmento em Pesquisa e Desenvolvimento. É dentro do contexto dessa Lei que o Ocean retira grande parte dos recursos de subsidiação do laboratório. 

Ainda, por estar dentro da universidade, a Samsung têm acesso direto a especialistas com \textit{know how} restrito a poucas instituições no mundo e acesso a graduandos e pós-graduandos com potencial para futuros recrutamentos 

\subsection{Alunos}
\label{sec:con_alunos}

A presença de um laboratório como este também ajuda a fomentar a cultura de empreendedorismo dentro da universidade, pois deixa os alunos próximos ao desenvolvimento de software, uma das principais bases de criação de novas \textit{startups}, devido ao baixo custo de aprendizado e investimento e alto valor gerado no curto e médio prazo. Ainda, segundo \citeonline{entrepreneurship}, os estudantes de engenharia experienciam o empreendedorismo de 4 maneiras: 

\begin{enumerate}
\item Primeiro passo para o auto-aprendizado
\item Preparação para a vida no trabalho
\item Caminho para ser autônomo
\item Desenvolvimento de liderança e responsabilidade de um time
\end{enumerate}

\subsection{Usuários Externos}
\label{sec:con_usuarios}

Segundo a 27\textsuperscript{a} Pesquisa de Anual do uso de TI, realizada pela Fundação Getúlio Vargas (FGV), o número de smartphones em uso no Brasil gira atualmente em torno de 168 milhões de dispositivos. \cite{tifgv} Não obstante, além do alto número de smartphones, o Brasil também se mostra presente no mercado de outros dispositivos inteligentes, com previsão de movimentação de US\$4,1 bilhões no Brasil com IOT, segundo a assessoria de imprensa da IDC Brasil. \cite{idc}

É nesse cenário de alto crescimento do uso de novas tecnologias no Brasil que o mesmo se mostra como um grande mercado para produtos inerentes ao uso de dispositivos inteligentes, como aplicativos e games. Dentro desse contexto, jovens interessados pelo desenvolvimento desse mercado no país podem utilizar o Ocean para realizar diferentes cursos nessas áreas, desde aulas para iniciantes até cursos mais avançados.

Além do uso aplicado diretamente nessa área de dispositivos portáteis, a programação desenvolvida nessas atividades pode ser extendida para outras áreas de desenvolvimento, tornando os jovens mais capacitados para qualquer área tecnológica. Segundo a ONG Code.org, financiada por fundadores das maiores empresas de tecnologia do mundo como Mark Zuckerberg e Bill Gates, o número de empregos para programadores cresce exponencialmente, ao passo que o ensino de programação nas escolas não acompanha o mesmo ritmo, o que gerará uma falta de profissionais de TI em um futuro próximo. Juntamente a essa informação, o departamento de estatísticas de trabalho dos Estados Unidos (\textit{Bureau of Labor Statistics}) estima que o número de empregos para programadores dentro dos EUA diminuirá em até 8\%, pois mais profissionais deverão ser recrutados fora do país, devido ao baixo custo e a flexibilidade de trabalho remoto permitida pela programação. \cite{bls}

\section{Levantamento das necessidades} % (fold)
\label{sec:necessidades}

\section{Definição de processos} % (fold)
\label{sec:processos}

\section{Definição de indicadores} % (fold)
\label{sec:indicadores}