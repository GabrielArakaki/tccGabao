%!TEX root = index.tex

\chapter{Metodologia}

\section{Método}

\label{chap:metodologia}

\begin{figure}[h]
\caption{Metodologia Utilizada no Trabalho}
\centerline{\includegraphics[scale=0.5]{img/metodologia}}
\label{fig:metodologia}
\caption* {Fonte: Elaborado pelo próprio autor}
\end{figure}

Após conversas iniciais com membros da Samsung e do PRO, foram definidos os stakeholders do projeto Ocean, juntamente com as pessoas-chave de cada. Foram realizadas entrevistas semi-estruturadas e gravadas, de forma a permitir a liberdade para serem feitas perguntas não presentes na estrutura inicial, e assim ter uma conversa guiada com o entrevistado como se não fosse uma entrevista formal. Antes de cada entrevista eram preparadas perguntas relacionadas ao mapeamento e percepção das interações do \textit{stakeholder} com o Ocean segundo a visão do entrevistador, e para todos os entrevistados era perguntado "se a pessoa teria alguma sugestão para a melhoria do Laboratório".

Em específico para os cursistas dos cursos básicos, o método de entrevistas foi considerado menos eficiente pelo alto número de interações com os participantes através das aulas. Portanto, foram aproveitados os questionários respondidos nos últimos dois anos de cursos - totalizando um total de 5280 respostas - para extrair as informações necessárias para este trabalho. O modelo de questionário aplicado encontra-se em anexo. No caso, foram analisadas somente as respostas das questões analíticas: "O que mais o motivou nesse curso?", "O que você acha que pode ser melhorado?" e "Qual tema você gostaria que fosse abordado num próximo curso?".

De forma a analisar e extrair informações desses questionários, o autor optou por desenvolver um software próprio (https://github.com/GabrielArakaki/wordAnalysis) para manipular esses dados via palavras-chave, pelos motivos elencados a seguir.

\begin{description}
\item [Volume de Dados] A quantidade de dados é grande o suficiente para dificultar - porém não inviabilizar - a leitura individual de cada uma das respostas. Acredita-se que mesmo que se optasse pela leitura individual, ela deveria ser acompanhada de uma \textit{clusterização} das respostas em categorias, o que a análise via repetição de palavras-chave também permite encontrar. Em contrapartida, os dados são suficientemente pequenos a ponto de não ser viável buscar alternativas  de \textit{Machine Learning} existentes para tratar esses dados, pois o volume não seria suficiente para treinar a inteligência artificial utilizada.

\item [Proposta do Estudo] Embora exista um campo muito grande de possibilidades de tratar e manipular os dados presentes, existe a necessidade de seguir com os objetivos inicias do presente estudo, que diz respeito ao posicionamento do Ocean de forma holística, e não somente na questão do aprendizado transmitido através de seus cursos. O autor acredita que há espaço para análises mais sofisticadas - dignas de um trabalho dedicado somente a isso - que poderiam ser realizadas caso a Samsung encontre a necessidade de entender os cursistas nos mínimos detalhes.

\item [Manipulação de Dados] O uso de uma ferramenta própria dá ao autor a liberdade de trabalhar e iterar em cima dos dados de forma a otimizar a ferramenta para o próprio uso. Existem disponíveis ferramentas prontas de geração de nuvens de palavras, que possuem um intuito muito maior de apresentar um 'choque visual' do que apresentar dados analíticos em si. A manipulação de dados permite que o autor encontre associações entre palavras ao observar os dados gerados pela ferramenta e trabalhar em cima da própria ferramenta para eliminar redundâncias ou dados inconsistentes.

\item [Princípio do Desenvolvedor] O princípio do desenvolvedor diz que se não há uma ferramenta própria que sirva para sua necessidade, desenvolva uma ferramenta que sirva. No caso, além de satisfazer com maior necessidade do que outras ferramentas presentes, é de satisfação do autor como desenvolvedor utilizar da programação para resolver problemas reais.
 
\end{description}

Vale destacar que os insights obtidos a partir desses modelos analíticos também são de grande valia para a Samsung, pois a parte qualitativa dos questionários nunca havia sido explorada anteriormente. 

Já para usuários de cursos intensivos, foram feitas entrevistas com representantes de \textit{startups} que frequentaram o curso anteriormente, centradas na pergunta: "Como a participação no curso intensivo mudou a vida da empresa e as vidas dos participantes?".

Para consolidar a análise realizada para cada \textit{stakeholder}, foi utilizada uma adaptação do modelo SWOT para ilustrar as percepções obtidas de cada um. Ela difere do modelo original do SWOT por não se tratar de uma análise de negócio baseada em fatores de mercado e competitividade com outros \textit{players}, e sim em uma análise fria e absoluta de fatores básicos em um projeto: Pontos Fortes (\textit{Strengths}), Pontos Fracos (\textit{Weaknesses}), Oportunidades (\textit{Opportunities}) e Ameaças (\textit{Threats}). Foi considerado que por ser um modelo simples e de fácil visualização, consolidaria as necessidades deste trabalho sem desvirtuar os objetivos propostos.

\begin{figure}[h]
\caption{Quadro SWOT básico}
\centerline{\includegraphics[scale=0.5]{img/detailedswot}}
\label{fig:detailedswot}
\caption* {Fonte: Quadro SWOT básico}
\end{figure}

A partir dos resultados obtidos pela análise das respostas, trabalhou-se em cima da geração de propostas de melhoria.

\section{Objeto de Estudo}

De forma a se aproximar de um dos principais nichos de seu interesse, a Samsung criou um programa de Relacionamento com Desenvolvedores, de forma a se aproximar de um grupo de profissionais que mais contribuem com a disseminação de novas tecnologias da Samsung. Esse programa possui diversas frentes, como o \textit{Developer Day}, que é um grande evento realizado uma vez por ano no Brasil e outros países da América Latina que visa promover as mais recentes tecnologias da marca, e o Laboratório Ocean, que visa a aproximação da comunidade estudantil e de startups em formação.

O Laboratório Ocean é uma iniciativa da Samsung que consiste em estimular desenvolvedores a criar soluções tecnológicas relacionadas aos produtos da marca coreana. A primeira sede do laboratório foi inaugurada em 2010 na Coréia do Sul, e a iniciativa foi replicada no Brasil há cerca de dois anos, com uma unidade em Manaus e outra em São Paulo. Ao passo que a unidade de Manaus foi estabelecida dentro da Universidade Estadual do Amazonas (UEA), a unidade de São Paulo encontrava-se até o fim de 2015 na Avenida Brigadeiro Faria Lima, uma das principais avenidas comerciais da cidade. Uma iniciativa recente movida por um ex-aluno, professores do departamento e o programa 'Parceiros da Poli' trouxe através de conversas informais a ideia de trazer o laboratório para dentro da USP. Como o modelo intra universitário funcionou bem em Manaus, foi decidido replicar o modelo e sediar o laboratório dentro da Universidade, hospedado dentro do Departamento de Engenharia de Produção (PRO).

O Ocean fornece dois tipos de cursos, básicos e intensivos. Os cursos básicos são de curta duração (aproximadamente 3 horas), e os cursos intensivos em seu módulo atual duram 4 meses, utilizando o espaço toda noite de segunda à quinta-feira. O foco inicial dos cursos foi o desenvolvimento em dispositivos móveis, em especial apoiados no sistema operacional Android, inerente aos aparelhos da Samsung, como o Galaxy S7. Com o passar do tempo, os cursos começaram a seguir as tendências de \textit{hardware} do mercado e consequentemente da própria Samsung, como \textit{wearables}, \textit{smart} TVs, Internet das Coisas e Realidade Virtual. Mesmo assim, a área de dispositivos móveis ainda representa 80\% dos cursos oferecidos por eles.

Os cursos curtos possuem como principal objetivo despertar o interesse de desenvolvedores em relação aos produtos da Samsung. Portanto, os cursos trabalham de forma a mostrar todos os produtos de alta tecnologia da samsung e capacitar desenvolvedores para que utilizem os seus dispositivos através do desenvolvimento de softwares. Para tal, é disseminado tanto o funcionamento dos \textit{Software Development Kits} (SDKs) da Samsung e suas APIs para permitir o acesso ao \textit{hardware} dos seus dispositivos quanto o uso do Android para manipulação do software na linguagem nativa atual do sistema operacional utilizado por eles. Para a execução desses cursos, a Samsung trabalha juntamente com empresas terceiras especialistas no assunto para preparar o material a ser passado. Algumas vezes funcionários da própria Samsung dão o treinamento, e em alguns momentos houve até participação do corpo docente da Poli.

Os cursos intensivos são cursos de pré-aceleração de empresas, e têm o intuito de fomentar o empreendedorismo, apesar de manter a base de disseminar o conhecimento em cima de produtos da Samsung. A empresa acredita que no atual mercado, a diferenciação competitiva sobre o \textit{hardware} está ficando cada vez mais difícil, por isso as empresas estão buscando se diferenciar frente às outras em conteúdo. Dentro desse contexto, a Samsung visa auxiliar empresas a se desenvolverem e elas - em contrapartida - auxiliam a enriquecer os produtos da Samsung, seja através de novos produtos ou através de serviços.

Dessa forma, os cursos de pré-aceleração procuram fornecer conhecimento e experiência ao desenvolvimento de suas empresas, através de mentorias, bate-papo, palestras e avaliações. Como são cursos gratuitos, um dos principais desafios é manter o próprio engajamento das empresas, por isso o motivo de haver encontros 4 vezes por semana, com mentoria, criação e gestão do projeto proposto pelo programa, com checkpoints de avaliação das empresas ao longo do projeto. Tudo isso feito de forma \textit{gamificada} dentro do próprio modelo. A primeira parte do programa consiste da validação do modelo de negócio proposto pela empresa, e a segunda parte corresponde à prototipação e desenvolvimento de produto de fato. Atualmente contribuem com esse programa os profissionais da Samsung, funcionários terceiros, professores da USP, membros do NEU e empresas parceiras (Sebrae, FIESP, IBM, Amazon).

A infraestrutura do laboratório consiste em uma grande sala para até 80 pessoas, porém caso necessário portas retráteis permitem a sua divisão em duas salas separadas. Essa estrutura fica aberta das 08 às 22 horas de segunda a sexta feira, podendo ser utilizada livremente pela comunidade estudantil da universidade, cedendo computadores e acesso a Wi-Fi de alta velocidade. As mesmas salas são utilizadas para a realização dos cursos mencionados anteriormente.

Por se tratar de um acordo entre a Samsung e o PRO, é necessário que sejam feitas reuniões de alinhamento das necessidades e expectativas entre partes, que não estão sendo realizadas nesse primeiro momento pois o projeto ainda está no início e não há conflitos aparentes. Entretanto a universidade também tem planos para o laboratório e acredita que o mesmo terá um grande impacto dentro e fora da universidade. Segundo as palavras do professor Eduardo Zancul na inauguração do Ocean: “É uma frente de ensino, pesquisa e extensão. Ensino pois será um espaço para disciplinas do curso de engenharia de produção; Pesquisa porque materiais e a estrutura do laboratório serão utilizados pela comunidade acadêmica; Extensão pois muitos cursos serão abertos para a comunidade”. O laboratório se tornou uma parceria de cogestão entre universidade e empresa que tem como principal mérito a geração de valor derivada da sinergia entre academia e indústria. 

\section{Identificação de Stakeholders do Laboratório}
\label{sec:identificacao_stakeholders}

De forma a ilustrar os principais stakeholders do laboratório representados nesse trabalho, foi criado um mapa representativos dos mesmos. 

\begin{figure}[h]
\caption{Mapa de stakeholders do projeto Ocean}
\centerline{\includegraphics[scale=0.5]{img/stakeholders_v2}}
\label{fig:stakeholders}
\caption* {Fonte: Elaborado pelo próprio autor}
\end{figure}

\subsection{PRO}
\label{sec:con_pro}

O Ocean é um dos quatro grandes projetos que o PRO acompanha atualmente, esquematizado pela Tabela \ref{tab:pilares_pro}, juntamente com Inovalab, a Fábrica Didática e o Núcleo de Empreendedorismo da USP.

\begin{table}[H]
\begin{center}
\caption{Pilares do PRO}
\label{tab:pilares_pro}
{\def\arraystretch{2}\tabcolsep=10pt
\begin{tabular}{>{\raggedright}p{0.2\linewidth}>{\raggedright\arraybackslash}p{0.2\linewidth}>{\raggedright\arraybackslash}p{0.2\linewidth}>{\raggedright\arraybackslash}p{0.2\linewidth}}
\hline
     & Objetivo Institucional & Participação do PRO & Em Atividade  \\ \hline
     Inovalab & Laboratório de Inovação & Gestão Ativa & Sim  \\
     Fábrica Didática & Apropriação de conceitos de fábricas para aplicação no ensino & Em Desenvolvimento & Não \\
     Ocean & Laboratório de Desenvolvimento de Software & Cogestão com a Samsung & Sim \\
	 Núcleo de Empreendedorismo da USP & Disseminação da cultura empreendedora & Cede espaço físico & Sim \\ \hline
\end{tabular}%
}
\caption* {Fonte: Elaborado pelo autor em conversa com professores do departamento}
\end{center}
\end{table}

O Inovalab é um laboratório que oferece recursos para realização de projetos de engenharia, como \textit{software}, \textit{hardware}, impressoras 3D e oficinas mecânicas. A infraestrutura do laboratório é utilizada para sediar o NEU, outro dos projetos que será discutido adiante. Já a Fábrica Didática é um projeto que consiste em uma unidade de ensino e pesquisa com foco em fabricação e tecnologia.

O que pode se ver em comum entre esses projetos e o Ocean é a proximidade da inovação e do empreendedorismo, pontos que o PRO considera essenciais para a tríade pesquisa, ensino e extensão. Segundo o Professor Dr. Fernando Laurindo, a Poli propicia um ambiente de aprendizado com disciplinas que já contribuem nessa área, como Projeto Integrado e Desenvolvimento de Produto, porém grande parte do aprendizado pode ser obtido através de atividades extra curriculares, propiciadas em parte por esses projetos. Um dos principais métodos que a Poli incentiva nos alunos é o \textit{self-learning}, pois o aluno que deve buscar profundidade em certos temas, e a Poli fornece a estrutura para que ele possa absorver conhecimento com facilidade.

O empreendedorismo, em particular, sempre acompanhou a trajetória de alunos da Poli, que optaram por abrir a sua própria empresa ao invés de entrarem no mercado de trabalho. Muitas empresas tiveram o desenvolvimento de seu modelo de negócio em paralelo com o trabalho de graduação de alunos. Por isso a Poli busca sempre oferecer recursos que deem a estrutura necessária para os alunos buscarem o seu aprendizado.

\subsection{NEU}
\label{sec:con_neu}

O Núcleo de Empreendedorismo da USP é uma organização formada por alunos de graduação e pós-graduação, pesquisadores e professores que possuem a missão de promover a cultura de empreendedorismo dentro da Universidade. O NEU é aberto à toda comunidade da USP, já tendo recebido contribuições de diversas instituições da universidade, porém é formado atualmente principalmente por alunos da POLI e da FAU.

De forma similar ao estabelecimento do Ocean dentro do departamento do PRO, o NEU foi convidado a utilizar o espaço do Laboratório de Inovação (InovaLab) para sediar suas atividades. Atualmente o NEU trabalha com três principais pilares: inspiração, capacitação e conexão.

Inspiração diz respeito ao fomento ao empreendedorismo nos alunos para que eles se sintam impulsionados a participar do ecossistema de \textit{startups} ou até abrir as suas próprias. Portanto são feitos diversos convites aos diretores de diversas \textit{startups}, muitos com origens da própria USP, como Lean Survey, 99 Táxis e Squid, e estes podem explicar um pouco da sua trajetória e das emoções vividas graças aos seus empreendimentos. 

Capacitação é a frente do NEU de auxiliar ideias de alunos a se desenvolverem em produtos, para que assim sejam criadas novas empresas. A partir da rede de empresas que o NEU tem em seu leque de contatos, ele consegue encontrar mentorias para as empresas e acelerar o seu desenvolvimento. O principal programa dessa frente é o \textit{Startup Lab}, em que o NEU fornece material de apoio e mentoria através dos seus contatos com empresas, investidores e aceleradoras.

Conexão é representado principalmente pelo \textit{Startup Ship}, que é o canal do NEU destinado a alunos que querem estagiar em \textit{startups}. Através de sua rede de conexões ela facilita com que \textit{startups} e os alunos certos cheguem uns aos outros. Outro programa é o Pesquisas USP, que auxilia os alunos a se conectarem com pesquisas, e em contrapartida auxiliar pesquisadores a se conectarem com alunos ou empresas que possam auxiliar nos seus estudos. Nesse programa o NEU também auxilia startups a entrarem em contato com aceleradoras.

O NEU apresenta uma sinergia muito grande com o Ocean, pois ambos possuem muito interesse nessa fase de pré-aceleração de empresas, conseguindo exercer etapas distintas e complementares nesse processo. Durante os cursos intensivos do Ocean, o NEU se responsabiliza por trazer contatos de diferentes empresas para inspirar e fazer mentorias, ao passo que a Samsung trabalha fortemente na parte de capacitação e acompanhamento da evolução das empresas ao longo do programa.

Existe uma série de organismos dentro da universidade que fomentam a cultura de empreendedorismo, e como todos são gratuitos, existe uma colaboração muito grande para que os maiores beneficiados sejam as \textit{startups}, independente de  qual instituição que esteja contribuindo mais para a evolução da empresa.

\subsection{Alunos}
\label{sec:con_alunos}

A presença de um laboratório como este também ajuda a fomentar a cultura de empreendedorismo dentro da universidade, pois deixa os alunos próximos ao desenvolvimento de software, uma das principais bases de criação de novas \textit{startups}, devido ao baixo custo de aprendizado e investimento e alto valor gerado no curto e médio prazo. Ainda, segundo \citeonline{entrepreneurship}, os estudantes de engenharia experienciam o empreendedorismo de 4 maneiras: 

\begin{enumerate}
\item Primeiro passo para o auto-aprendizado
\item Preparação para a vida no trabalho
\item Caminho para ser autônomo
\item Desenvolvimento de liderança e responsabilidade de um time
\end{enumerate}

\subsection{Cursistas}
\label{sec:con_cursistas}

Conforme explicado anteriormente, o Ocean trabalha com duas propostas de cursos, os cursos básicos e os cursos intensivos. O primeiro modelo é de capacitação de desenvolvedores a trabalharem com dispositivos da Samsung, e o segundo modelo é para empreendedores que querem melhorar o modelo de negócio das suas propostas. Embora exista uma gama de pessoas que seja usuária de ambos os cursos - novos empreendedores têm muito interesse por programacão - o público-alvo de cada é diferente.

O primeiro modelo é desenhado para programadores ou aspirantes à programação. Embora existam diversos temas de cursos, o objetivo da Samsung é sempre de capacitar os cursistas a explorarem todas as capacidades permitidas pelo \textit{hardware} da Samsung. Para isso, serão explorados o sistema operacional Android, os SDKs da Samsung, o sistema operacional Tizen, em suma tudo que possa contribuir para o objetivo final. Portanto, é um curso de bastante interesse para todo tipo de pessoa que possui o desejo de desenvolver conteúdo que possa ser rodado em dispositivos da Samsung.

Segundo a 27\textsuperscript{a} Pesquisa de Anual do uso de TI, realizada pela Fundação Getúlio Vargas (FGV), o número de smartphones em uso no Brasil gira atualmente em torno de 168 milhões de dispositivos. \cite{tifgv}. Não obstante, além do alto número de smartphones, o Brasil também se mostra presente no mercado de outros dispositivos inteligentes, com previsão de movimentação de US\$4,1 bilhões no Brasil com IOT, segundo a assessoria de imprensa da IDC Brasil. \cite{idc}. Além do uso aplicado diretamente nessa área de dispositivos portáteis, a programação desenvolvida nessas atividades pode ser extendida para outras áreas de desenvolvimento, tornando os jovens mais capacitados para qualquer área tecnológica. Segundo a ONG Code.org, financiada por fundadores das maiores empresas de tecnologia do mundo como Mark Zuckerberg e Bill Gates, o número de empregos para programadores cresce exponencialmente, ao passo que o ensino de programação nas escolas não acompanha o mesmo ritmo, o que gerará uma falta de profissionais de TI em um futuro próximo. Juntamente a essa informação, o departamento de estatísticas de trabalho dos Estados Unidos (\textit{Bureau of Labor Statistics}) estima que o número de empregos para programadores dentro dos EUA diminuirá em até 8\%, pois mais profissionais deverão ser recrutados fora do país, devido ao baixo custo e a flexibilidade de trabalho remoto permitida pela programação. \cite{bls}

É nesse cenário de alto crescimento do uso de novas tecnologias no Brasil que o mesmo se mostra como um grande mercado para produtos inerentes ao uso de dispositivos inteligentes, como aplicativos e games. Dentro desse contexto, jovens interessados pelo desenvolvimento desse mercado no país se interessam por propostas como a do Ocean para realizar diferentes cursos nessas áreas.

Já em relação ao curso intensivo, o mesmo foi feito para incentivar ideias a se tornarem empresas, então é muito voltado para o nicho empreendedor, sem desapegar da área de tecnologia, que é de onde vêm toda \textit{expertise} da Samsung. Segundo \citeonline{empreendedorismo}, grande parte desse nicho empreendedor virá de universitários com as três principais características:

\begin{itemize}
\item Propensão a assumir riscos
\item Proximidade a outros empreendedores no círculo próximo
\item Já possuem uma ideia desenvolvida para o empreendimento
\end{itemize}

Logo, devido à localização e divulgação dentro da universidade, a maior parte dos cursistas do módulo intensivo virão de pessoas que possuem alguma laço com a universidade, como alunos e ex-alunos.