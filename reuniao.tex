%!TEX root = index.tex
\chapter[Reunião com a Click Bus]{Reunião com a Click Bus}
\label{chap:Reuniao_com_a_click_bus}

Desde o início do projeto tínhamos a intenção de trabalhar em parceria com uma empresa para que nosso trabalho fosse implementado para que pudéssemos mensurar sua efetividade em um ambiente real. Para isso fizemos contato com a Click Bus, uma loja online que vende passagens de ônibus, hoje a maior do país.

Na reunião que tivemos foi discutido o que nos seria fornecido para o desenvolvimento do projeto e o que deveríamos devolver ao final. Foi acordado que teríamos acesso ao banco de dados com alguns dos dados codificados para preservar a identidade de seus clientes e para que estes dados não fossem usados por outras empresas.

Nos foi mostrada a estrutura do banco de dados usado na empresa. O banco de dados está atualmente dividido em  seis diferentes tabelas:

\begin{\begin{itemize}
	\item Ordem
	\item Itens da ordem
	\item Rotas
	\item Lugares
	\item Pagamentos de ordens
	\item Endereço de cobrança
\end{itemize}

A tabela Ordem é composta por 7 campos:

\begin{itemize}
	\item idOrdem, o identificador único da ordem de compra
	\item Total Reais, o valor gasto na ordem
	\item idUsuario, neste caso o email do usuario
	\item nome, nome do usuario
	\item Criado em, data em que o pedido foi feito
	\item Status da ordem, com relação a pagamento e processamento do pedido
	\item Dia da semana, dia da semana em que a ordem foi feita
\end{itemize}

Itens da ordem é composta por 10 campos: 

\begin{itemize}
	\item idItens da ordem, é o identificador único do item
	\item Ordem_idOrdem, identificador de a qual ordem o item pertence, referencia a tabela Ordem
	\item Origem, cidade de origem
	\item Destino, cidade de destino
	\item Serviço, Leito, executivo ou comum
	\item idLinha de Onibus, qual a empresa que opera a viagem
	\item Departure date_time, dia e hora de saída
	\item Arrival date_time, dia e hora de chegada
	\item Ida/Volta, viagem de ida ou de volta
	\item Rotas_idRotas, identificador da rota a ser realizada, referencia a tabela Rotas
\end{itemize}

Rotas é composta de 3 itens:

\begin{itemize}
	\item idRotas, identificador único de rota
	\item Lugares_idLugares_Origem, Lugar de origem, referencia a tabela Lugares
	\item Lugares_idLugares_Destino, Lugar de destino, referencia a tabela Lugares
\end{itemize}

Lugares é composta por 3 itens:

\begin{itemize}
	\item  idLugares, identificador único de lugar
	\item Cidade, casos especiais ocorrem quando a cidade possui mais de uma estação rodoviária
	\item Estado
\end{itemize}

Pagamentos de ordens, 3 itens:

\begin{itemize}
	\item idPagamentos de ordens, identificador único de pagamento
	\item Ordem_idOrdem, referencia a qual ordem pertence o pagamento
	\item Tipo de pagamento, modo de pagamento, cartão de crédito etc.
\end{itemize}

A última tabela, Endereço de cobra consiste em 6 campos:
\begin{itemize}
	\item idEndereço de cobrança, identificador de cobrança
	\item Pagamentos de ordens_idPagamentos de ordem, referencia a qual pagamento o endereço pertence
	\item Cidade
	\item Estado
	\item Endereço
	\item Cep 
\end{itemize}

Os relacionamentos entre tabelas estão demonstrados na figura abaixo.

Inserir Figura aqui

Outro importante tópico conversado em reunião foi a questão da importância do timing da recomendação. O sistema deve não só prever qual será a próxima viagem do usuário, mas quando esta se dará e quanto tempo antes da viagem devemos oferecê-la ao cliente. Foi feito um estudo sobre quando os cliente costumam comprar as passagens pela Click Bus e nos foi informado que 60 por cento das passagens são compradas 3 dias antes da data de embarque. Então é importante que a sugestão seja feita antes que o cliente compre a passagem, porém perto o bastante da data de embarque para que esse tenha interesse.
