\chapter{Objeto}

\section{Laboratório Ocean}

O Laboratório Ocean é um programa da Samsung que consiste em estimular desenvolvedores a criar soluções tecnológicas relacionadas aos seus produtos. A primeira sede do laboratório foi inaugurada em 2010 na Coréia do Sul, e a iniciativa foi replicada no Brasil há cerca de dois anos, com uma unidade em Manaus e outra em São Paulo.

Diferentemente do laboratório de Manaus, que se encontra dentro da Universidade Estadual do Amazonas (UEA), a unidade de São Paulo encontrava-se na Avenida Brigadeiro Faria Lima, uma das principais avenidas comerciais da cidade, porém distante da comunidade acadêmica/estudantil. Uma iniciativa recente movida por um ex-aluno, professores do departamento e o programa 'Parceiros da Poli' trouxeram o laboratório para dentro da Universidade, dentro do Departamento de Engenharia de Produção (PRO).

Segundo as palavras do professor Eduardo Zancul na inauguração do Ocean: “É uma frente de pesquisa, ensino e extensão. Ensino pois será um espaço para disciplinas do curso de engenharia de produção; Pesquisa porque materiais do laboratório serão utilizados pela comunidade acadêmica; Extensão pois cursos serão abertos para a comunidade”. O laboratório se tornou uma parceria de cogestão entre universidade e empresa que tem como principal mérito a geração de valor derivada da sinergia entre as pesquisas da academia e o conhecimento aplicado da indústria. 

O Ocean é um dos quatro grandes projetos que o PRO acompanha atualmente, esquematizado pela Tabela \ref{tab:pilares_pro}.

\begin{table}[h]
\begin{center}
\caption{Pilares do PRO}
\label{tab:pilares_pro}
{\def\arraystretch{2}\tabcolsep=10pt
\begin{tabular}{>{\raggedright}p{0.2\linewidth}>{\raggedright\arraybackslash}p{0.2\linewidth}>{\raggedright\arraybackslash}p{0.2\linewidth}>{\raggedright\arraybackslash}p{0.2\linewidth}}
\hline
     & Objetivo Institucional & Participação do PRO & Em Atividade  \\ \hline
     Inovalab & Laboratório de Inovação & Gestão Ativa & Sim  \\
     Fábrica Didática & Apropriação de conceitos de fábricas para aplicação no ensino & Em Desenvolvimento & Não \\
     Ocean & Laboratório de Desenvolvimento de Software & Cogestão com a Samsung & Sim \\
	 Núcleo de Empreendedorismo da USP & Disseminação da cultura empreendedora & Cede espaço físico & Sim \\ \hline
\end{tabular}%
}
\caption* {Fonte: Elaborado pelo autor em conversa com professores do departamento}
\end{center}
\end{table}

Portanto, é de grande interesse do PRO garantir que o laboratório esteja sendo utilizado da melhor forma possível, sendo necessário mapear todos os interessados pelo bom funcionamento e traçar um plano de ações e indicadores de forma a garantir o pleno potencial do laboratório.

//Colocar informações sobre estrutura de gestão, pontos de contato