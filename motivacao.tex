%!TEX root = index.tex
\section{Motivação} % (fold)
\label{cha:motivacao}

Como aluno atuante no mercado de tecnologia, o presente autor vivencia na prática a deficiência de comunicação entre o mercado e a comunidade científica-acadêmica, gerando um \textit{gap} entre as demandas das empresas e o conteúdo ensinado nas aulas. Devido ao período de grande evolução exponencial da tecnologia das últimas décadas, é necessário que a comunidade acadêmica e as principais escolas de ensino acompanhem essa evolução oferecendo cursos intra e extra curriculares que acompanhem essas tendências.

Nesse contexto, encontra-se a programação como uma das principais necessidades de ensino, pois esta é a base do funcionamento de grande parte das empresas e \textit{startups} atuais. É muito importante que os futuros gestores dessas empresas entendam o funcionamento dessa operação de "alto nível" para otimizar os processos, fazer uma melhor gestão de projetos e conseguir identificar possíveis gargalos no sistema.

Desta maneira, o laboratório Ocean se mostra não apenas como uma parceria entre empresa e universidade para P\&D e Inovação, mas como uma fonte de \textit{feedbacks} constantes sobre as tendências e necessidades de cada um, que devem ser utilizados para fortalecer a universidade em seus pilares: pesquisa, ensino e extensão.

