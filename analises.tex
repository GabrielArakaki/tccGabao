%!TEX root = index.tex
\chapter{Analises}
\label{cha:analises}

\section{Samsung}

A partir de conversas com Luís Guilherme Selber, responsável pela operação do Laboratório Ocean, obtivemos a seguinte análise:

\begin{figure}[H]
\caption{Análise do Ocean - Samsung}
\centerline{\includegraphics[scale=0.75]{img/samsungswot}}
\label{fig:swotsamsung}
\caption* {Fonte: Elaborado pelo próprio autor}
\end{figure}

A ausência de fraquezas e ameaças se dá - segundo o entrevistado - pelo fato de o projeto estar em uma fase inicial. Não há formalidades estabelecidas como reuniões periódicas, quando é necessário conversar sobre algo é fácil encontrar um professor no corredor ou em sua sala, portanto as interações se limitam às necessidades de um ou outro, sem conflitos até o momento. A própria gestão de uso do Laboratório para as atividades de cada instituição não gera conflitos pois as necessidades de uso do laboratório por cada parte já foram claramente estabelecidas nas reuniões iniciais de fechamento do projeto.

Em relação aos pontos fortes, destaca-se o fato de o laboratório sofrer uma mudança muito positiva devido à mudança de localidade para dentro da universidade. A presença do laboratório dentro do PRO mudou completamente a utilização do laboratório, que agora é ocupado das 08 da manhã até as 22 da noite de segunda a sexta, fato que não acontecia na Faria Lima. Quando na Av. Brigadeiro Faria Lima, a Samsung exercia um papel ativo de sediar eventos próprios ou externos com o intuito de divulgar e preencher o espaço do laboratório. Tudo isso porque mais pessoas utilizando o espaço ajuda a divulgar mais o programa e os cursos oferecidos pelo laboratório. Não obstante, além dos próprios cursos dados pelo laboratório, ainda existe um custo com equipamentos e internet que não deveria existir em vão.

Outra grande vantagem de estar dentro da universidade consiste no fato de a USP ser um ecossistema de parcerias com diversas empresas. Dentro desse meio, a Samsung ganha expande sua rede de contatos com outras empresas, já tendo provido até o momento reuniões com outras grandes empresas do mercado nacional.

O último ponto forte mencionado foi o aumento de demanda pelos cursos básicos e intensivos, houve um grande aumento principalmente por alunos da universidade. A proximidade com o NEU e a POLI permite uma divulgação muito mais fácil do laboratório na universidade.

Em relação às oportunidades, a Samsung vê como maiores oportunidades de melhoria uma maior participação de professores nos cursos do laboratório, seja através de aulas ou mentorias. Acredita-se que exista uma sinergia muito bom entre a academia e empresa no sentido de prender a atenção e incentivar os alunos dos cursos a buscarem mais conhecimento nas suas áreas de interesse.

Também acredita-se que possa haver uma adesão maior aos cursos e ao próprio dia a dia do laboratório por parte de outras instituições da Universidade. Tanto a FEA quanto o IME por suas formações ligadas ao empreendedorismo e ao desenvolvimento mostram ser muito compatíveis com os programas oferecidos pelo Ocean, entretanto a grande maioria de adeptos vêm da Escola Politécnica.

\section{PRO}

\begin{figure}[H]
\caption{Análise do Ocean - PRO}
\centerline{\includegraphics[scale=0.75]{img/generalswot}}
\label{fig:swotpro}
\caption* {Fonte: Elaborado pelo próprio autor}
\end{figure}

\section{NEU}

\begin{figure}[H]
\caption{Análise do Ocean - NEU}
\centerline{\includegraphics[scale=0.75]{img/generalswot}}
\label{fig:swotneu}
\caption* {Fonte: Elaborado pelo próprio autor}
\end{figure}

\section{Alunos}

\begin{figure}[H]
\caption{Análise do Ocean - Alunos}
\centerline{\includegraphics[scale=0.75]{img/generalswot}}
\label{fig:swotalunos}
\caption* {Fonte: Elaborado pelo próprio autor}
\end{figure}

\section{Usuários Externos}

\begin{figure}[h]
\caption{Análise do Ocean - Usuários Externos}
\centerline{\includegraphics[scale=0.75]{img/generalswot}}
\label{fig:swotusuarios}
\caption* {Fonte: Elaborado pelo próprio autor}
\end{figure}