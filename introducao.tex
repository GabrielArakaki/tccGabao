%!TEX root = index.tex
\chapter[Introdução]{Introdução}
\label{chap:introducao}
\section{Contextualização do trabalho}
\label{cha:contexto}

O Laboratório Ocean é um programa da Samsung que consiste em estimular desenvolvedores a criar soluções tecnológicas relacionadas aos seus produtos. A primeira sede do laboratório foi inaugurada em 2010 na Coréia do Sul, e a iniciativa foi replicada no Brasil há cerca de dois anos, com uma unidade em Manaus e outra em São Paulo.

Diferentemente do laboratório de Manaus, que se encontra dentro da Universidade Estadual do Amazonas (UEA), a unidade de São Paulo encontrava-se na Avenida Brigadeiro Faria Lima, uma das principais avenidas comerciais da cidade, porém distante da comunidade acadêmica/estudantil. Uma iniciativa recente movida por um ex-aluno, professores do departamento e o programa 'Parceiros da Poli' trouxeram o laboratório para dentro da Universidade, dentro do Departamento de Engenharia de Produção (PRO).

Segundo as palavras do professor Eduardo Zancul na inauguração do Ocean: “É uma frente de pesquisa, ensino e extensão. Ensino pois será um espaço para disciplinas do curso de engenharia de produção, pesquisa porque materiais do laboratório serão utilizados pela comunidade acadêmica e extensão pois cursos serão abertos para a comunidade”. O laboratório se tornou uma parceria de cogestão entre universidade e empresa que tem como principal mérito a geração de valor derivada da sinergia entre as pesquisas da academia e o conhecimento aplicado da indústria. 

O Ocean é um dos quatro grandes projetos que o PRO acompanha atualmente, esquematizado pela figura a seguir: (DESENHAR FIGURA QUE SERA COLOCADA AQUI)

\begin{itemize}
\item Inovalab, que existe e funciona, e é interno ao PRO
\item Fábrica Didática, que não existe mas seria interno ao PRo
\item Ocean, que existe e é de cogestão PRO Samsung
\item NEU, que existe e funciona mas é externo ao PRO
\end{itemize}

Portanto, é de grande interesse do PRO garantir que o laboratório esteja sendo utilizado da melhor forma possível, sendo necessário mapear todos os interessados pelo bom funcionamento e traçar um plano de ações e indicadores de forma a garantir o pleno potencial do laboratório.

\section{Motivação} % (fold)
\label{cha:motivacao}

Como aluno atuante no mercado de tecnologia, o presente autor vivencia na prática a deficiência de comunicação entre o mercado e a comunidade científica-acadêmica, gerando um \textit{gap} entre as demandas das empresas e o conteúdo ensinado nas aulas. Devido ao período de grande evolução exponencial da tecnologia das últimas décadas, é necessário que a comunidade acadêmica e as principais escolas de ensino acompanhem essa evolução oferecendo cursos intra e extra curriculares que acompanhem essas tendências.

Nesse contexto, encontra-se a programação como uma das principais necessidades de ensino, pois esta é a base do funcionamento de grande parte das empresas e \textit{startups} atuais. É muito importante que os futuros gestores dessas empresas entendam a empresa a nível operacional de forma a otimizar os processos, fazer uma melhor gestão de projetos e conseguir identificar possíveis gargalos no sistema.

Desta maneira, o laboratório Ocean se mostra não apenas como uma parceria entre empresa e universidade nas frentes de  P\&D e Inovação, mas como uma fonte de \textit{feedbacks} constantes sobre as tendências e necessidades de cada um, que devem ser utilizados para fortalecer a universidade em seus pilares: pesquisa, ensino e extensão.

\section[Objetivos]{Objetivos}
\label{chap:objetivos}

O objetivo deste Trabalho de Conclusão de Curso é estabelecer processos de acompanhamento e avaliação e KPIs para avaliar se as necessidades de cada um dos \textit{stakeholders} do Laboratório Ocean está sendo atendida, de forma a garantir uma operação sustentável do laboratório.