%!TEX root = index.tex
\chapter[Introdução]{Introdução}
\label{chap:introducao}

Sob a óptica da sociedade, a universidade é frequentemente analisada apenas pela sua capacidade de formar profissionais com boas competências para atuar no mercado de trabalho. Embora o ensino e capacitação possa ser considerado o principal objetivo da universidade, a unicidade desse ponto de vista acaba por omitir todos os outros papéis que ela exerce, como pesquisa e os serviços à comunidade, além de subestimar toda sua complexidade diante da vasta quantidade de interações com agentes externos, necessárias para que ela cumpra todas as suas funções.

Para que a universidade possa atuar de forma plena e maximize os resultados diante da sociedade, é necessário que tanto o Governo quanto a Indústria participem e colaborem ativamente com a Universidade. O primeiro é responsável principalmente pela regulamentação do ensino, pela infraestrutura e pelo incentivo financeiro das universidades. Já último representa o mercado de trabalho, as demandas de recursos humanos e tecnológicos das empresas, e são os principais balizadores do ensino e da pesquisa gerados na universidade. 

A situação econômica atual do país fornece um contexto muito bom para ilustrar a relação entre a tríade Universidade, Governo e Indústria. Em época de forte crise financeira e alta inflação, o consumo de bens e serviços é desestimulado, intensificando a própria crise e gerando alguns problemas, como a diminuição de repasses financeiros do Governo para as instituições públicas. Para as instituições de ensino, o Imposto sobre Circulação de Mercadorias e Serviços (ICMS) se apresenta como a principal fonte de financiamento do ensino superior público paulista, portanto a diminuição do consumo reflete diretamente na diminuição do investimento que é feito nas universidades. Cabe à universidade buscar parcerias com empresas privadas para viabilizar a realização de novos projetos.

O presente trabalho foca na interação entre Universidade e Indústria, apresentada como Interação Universidade-Empresa (IUE). O papel e atuação do Governo será descrito diversas vezes ao longo do trabalho, porém não exerce papel ativo no atual objeto de estudo.

\section{Objeto de Estudo}

A IUE deste trabalho é representada pela Escola Politécnica da USP (POLI) e a Samsung, através do laboratório Ocean, que possui sede no departamento de Engenharia de Produção (PRO) da POLI e é administrado sob cogestão de ambas as partes. Dado o contexto atual da Universidade de forte fomento à inovação e ao empreendedorismo, o laboratório oferece a experiência de uma das maiores empresas do mundo em termos de inovação tecnológica aos seus alunos e à comunidade. Não obstante, a sua incorporação para dentro do departamento se apresenta como um investimento externo para dentro da universidade frente ao enfraquecimento do governo em seu papel de financiador.

O laboratório é uma iniciativa internacional da Samsung, e tem como principal objetivo estimular desenvolvedores e empreendedores a gerar conteúdo nas áreas \textit{mobile} e \textit{Internet of Things} (IOT), através da capacitação técnica e mentorias em relação ao modelo de negócio e desenvolvimento de produto. Ainda em fase inicial na POLI, o laboratório já é usado pelo corpo docente para o ensino de disciplinas do PRO, pela Samsung para cursos de desenvolvimento de aplicações e de pré-aceleração de empresas e pelos alunos como ambiente de estudos ou realização de projetos, porém ele ainda possui disponibilidade e estrutura para apoiar mais projetos que estão por vir.

\section{Justificativa}
\label{cha:justificativa}

Como aluno atuante no mercado de tecnologia, o presente autor vivencia na prática a deficiência de comunicação entre o mercado e a comunidade científico-acadêmica, gerando um \textit{gap} entre as demandas das empresas e o conteúdo ensinado nas aulas. Devido ao período de grande evolução exponencial da tecnologia das últimas décadas, é necessário que a comunidade acadêmica e as principais escolas de ensino acompanhem essa evolução oferecendo cursos intra e extra curriculares que acompanhem essas tendências.

Desta maneira, o laboratório Ocean se mostra não apenas como uma parceria entre empresa e universidade nas frentes de inovação e empreendedorismo, mas como uma fonte de \textit{feedbacks} em tempo real sobre as tendências de mercado e ensino, que devem ser comunicadas por ambos para obter mais resultados da parceria já existente.

Dentro desse contexto, encontra-se na programação e no desenvolvimento de produtos uma das principais necessidades de ensino, por ser a base do funcionamento de grande parte das empresas e \textit{startups} atuais. O autor considera que é muito importante que os futuros gestores de empresas entendam a empresa a nível operacional de forma a otimizar os processos, fazer uma melhor gestão de projetos e conseguir identificar possíveis gargalos no sistema.

Não obstante, a literatura atual sobre casos de IUE é muito voltada à frente de pesquisa e escassa em ensino e extensão, além de ser normalmente orientada somente pela óptica da universidade, com pouco acesso aos resultados e percepções das empresas parceiras diante dessa cooperação. O modelo de cogestão do Ocean mostra-se um bom candidato a gerar dados que cubram essa escassez, pois ele se baseia em uma interação contínua entre o PRO e a Samsung, gerando muita informação que pode ser útil à comunidade acadêmica. 

\section[Objetivos]{Objetivos}
\label{chap:objetivos}

Primeiramente, este trabalho busca identificar problemas e oportunidades de melhoria no funcionamento do Laboratório Ocean e oferecer uma proposta de priorização e de desenvolvimento dos pontos levantados de forma que auxiliem a sua gestão a priorizar as ações estratégicas a serem decididas e a operacionalizar futuros projetos, maximizando o uso do laboratório de forma sustentável.

Não obstante, este trabalho também tem como objetivo servir como guia para qualquer pessoa entender o funcionamento, gerenciamento e a estratégia do laboratório Ocean, seja o leitor um membro ativo da gestão, um pesquisador disposto a desenvolver novas pesquisas ou um funcionário da Samsung que deseja ter a visibilidade do projeto não só do ponto de vista da empresa. Ao ilustrar o funcionamento do Ocean como um grande projeto monolítico, o trabalho se sobrepõe a possíveis divisões de responsabilidades existentes na cogestão, oferecendo uma visibilidade única para as frentes de atuação e interação que são realizadas pelos programas do laboratório, independente de qual gestão é responsável por cada programa.

Dessa forma, este trabalho espera auxiliar a gestão do laboratório visando o curto e médio prazos, alinhando com a gestão as necessidades dos principais \textit{stakeholders} dessa parceria entre a Samsung e o PRO, de tal forma que possam ser investidos tempo e recursos nas melhores ações e assim obter o melhor uso do laboratório. 