%!TEX root = index.tex
\chapter[Introdução]{Introdução}
\label{chap:introducao}
\section{Contexto}
\label{cha:contexto}

Sob a óptica da sociedade, a universidade é normalmente analisada apenas pela sua capacidade de formar profissionais com boas competências para atuar no mercado de trabalho. Embora o ensino e capacitação seja talvez a tarefa mais importante da universidade, a unicidade desse ponto de vista acaba por omitir todos os outros papéis que ela exerce, além de subestimar toda sua complexidade diante da vasta quantidade de interações com os agentes necessários para que ela cumpra as suas funções.

Para que a universidade possa atuar de forma plena e maximize os resultados diante da sociedade, é necessário que tanto o Governo quanto a Indústria participem e colaborem ativamente com a Universidade. O primeiro é responsável principalmente pela regulamentação do ensino, pela infraestrutura e pelo incentivo financeiro das universidades. Já último representa o mercado de trabalho, as demandas de recursos humanos e tecnológicos das empresas, e são os principais balizadores do ensino e da pesquisa gerados na universidade.

O presente trabalho foca na interação entre Universidade e Indústria, apresentada como Interação Universidade-Empresa (IUE). O papel e atuação do Governo será descrito diversas vezes ao longo do trabalho, porém não faz parte do atual objeto de estudo.

\section{Objeto de Estudo}

Em época de forte crise financeira no país e alta inflação, o consumo de bens e serviços é desestimulado, intensificando a própria crise e gerando alguns outros problemas, como a diminuição de repasses financeiros do Governo para outras instituições. No contexto do presente trabalho, o Imposto sobre Circulação de Mercadorias e Serviços (ICMS) se apresenta como a  principal fonte de financiamento do ensino superior público paulista, e a sua queda reflete diretamente nos controle de custos da universidade.

Com o enfraquecimento do governo em seu papel de financiador e intermediador, é repassada à indústria exercer esse papel, e a IUE se intensifica com ambas as entidades atuando diretamente, com pouca intermediação do Governo.

A IUE deste trabalho é representada pela Escola Politécnica da USP (POLI) e a Samsung, através do laboratório Ocean, que possui sede no departamento de Engenharia de Produção (PRO) da POLI e é administrado sob cogestão de ambas as partes.

O laboratório é uma iniciativa internacional da Samsung, e tem como principal objetivo estimular desenvolvedores a criar diferentes soluções tecnológicas nas áreas \textit{mobile} e \textit{Internet of Things} (IOT). Ainda em fase inicial na POLI, o laboratório já é usado pelo corpo docente para lecionamento de disciplinas do PRO e pela Samsung para cursos de desenvolvimento de aplicações, porém o laboratório ainda possui disponibilidade e estrutura para um maior volume e variedade de fins.

\section{Justificativa}
\label{cha:justificativa}

Como aluno atuante no mercado de tecnologia, o presente autor vivencia na prática a deficiência de comunicação entre o mercado e a comunidade científico-acadêmica, gerando um \textit{gap} entre as demandas das empresas e o conteúdo ensinado nas aulas. Devido ao período de grande evolução exponencial da tecnologia das últimas décadas, é necessário que a comunidade acadêmica e as principais escolas de ensino acompanhem essa evolução oferecendo cursos intra e extra curriculares que acompanhem essas tendências.

Nesse contexto, encontra-se na programação uma das principais necessidades de ensino, por ser a base do funcionamento de grande parte das empresas e \textit{startups} atuais. O autor considera que é muito importante que os futuros gestores de empresas entendam a empresa a nível operacional de forma a otimizar os processos, fazer uma melhor gestão de projetos e conseguir identificar possíveis gargalos no sistema.

Desta maneira, o laboratório Ocean se mostra não apenas como uma parceria entre empresa e universidade nas frentes de  P\&D e Inovação, mas como uma fonte de \textit{feedbacks} constantes sobre as tendências e necessidades de cada um, devendo ser utilizados por ambos para melhorar a qualidade dos seus serviços e produtos.

Não obstante, a literatura atual sobre casos de IUE é muito voltada à frente de pesquisa e escassa em ensino e extensão, além de ser normalmente orientada somente pela óptica da universidade, com pouco acesso aos resultados e percepções das empresas parceiras diante dessa cooperação. O modelo de cogestão do Ocean mostra-se um bom candidato a gerar dados que cubram essa escassez, pois força uma interação contínua entre o PRO e a Samsung, gerando muitos \textit{feedbacks} que podem ser úteis à comunidade acadêmica. 

\section[Objetivos]{Objetivos}
\label{chap:objetivos}

O objetivo deste trabalho consiste em identificar oportunidades e melhorias na utilização do Laboratório Ocean, de forma que auxiliem a gestão do Ocean a maximizar o uso do laboratório de forma sustentável. Para tal, deve-se haver um esforço para manter todos os stakeholders do laboratório satisfeitos ao longo da sua operação, de forma a investirem tempo e recursos necessários para obter o melhor uso do laboratório.