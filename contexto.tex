%!TEX root = index.tex
\chapter{Contexto}
\label{cha:contexto}

\section{Identificação dos Stakeholders}
\label{sec:identificacao_stakeholders}

Ao longo da implementação do Ocean, foram definidos os seguintes stakeholders:

\begin{itemize}
\item PRO ( Chefia e COC [Zancul e Mesquita] e Pesquisa [Roberta])
\item Samsung (Selber, Conejo)
\item Alunos (CAEP, RD)
\item Usuários Externos ( Reinaldo [IGDA], Sakuda [Abragames], IoT)
\end{itemize}

\section{PRO}
\label{sec:con_pro}

Segundo \citeonline{jeeModels}, o ensino de engenharia pode ser dividido em 3 principais modelos: Acadêmico, Market-Driven e Integrativo.

\begin{table}[h]
\begin{center}
\caption{Modelos de ensino de engenharia}
\label{tab:modelos_ensino_tab}
{\def\arraystretch{2}\tabcolsep=10pt
\begin{tabular}{>{\raggedright}p{0.2\linewidth}>{\raggedright\arraybackslash}p{0.2\linewidth}>{\raggedright\arraybackslash}p{0.2\linewidth}>{\raggedright\arraybackslash}p{0.2\linewidth}}
\hline
     & Modelo Acadêmico & Modelo Market-Driven & Modelo Integrativo \\ \hline
     Percepção de Engenharia & Ciência Aplicada & Inovação Tecnológica & Serviço Público \\
     Papel Social & Consultor, Especialista & Empreendedor, Gestor & Cidadão, Agente de Mudanças \\
     Perspectiva Institucional & Universidade Científica & Universidade Empreendedora & Universidade Ecológica  \\
	 Disciplinas & Cálculo, Estatística & Empreendedorismo, Desenvolvimento de Produto & Sustentabilidade, Problemas da Sociedade \\ \hline
\end{tabular}%
}
\end{center}
\end{table}

Embora exista uma tendência das universidades em trabalhar principalmente com o primeiro modelo, ele não deixa de ser um modelo idealizado, portanto cada vez mais é gerado espaço para os outros modelos através de iniciativas da gestão do ensino. Aulas de Empreendedorismo, Marketing e Desenvolvimento de Produto representam muito bem um modelo Market-Driven, ao passo que aulas de Sustentabilidade e Desenvolvimento de Problemas da sociedade representam bem o modelo Integrativo.

No caso do modelo \textit{market-driven}, são abertas as portas para estruturas mais práticas de ensino, e dentro desse contexto observa-se um papel fundamental do Ocean para o departamento: auxiliar no desenvolvimento de futuros engenheiros para estarem alinhados com as necessidades do mercado. 

\section{Samsung}
\label{sec:con_samsung}

\begin{itemize}
\item Lei de Incentivo
\item Pesquisa
\item Recrutamento
\end{itemize}

\section{Alunos}
\label{sec:con_alunos}

\begin{itemize}
\item Empreendedorismo
\item Ensino
\item Extra-curricular
\end{itemize}

\section{Usuários Externos}
\label{sec:con_usuarios}

\begin{itemize}
\item Empreendedorismo
\item Ensino
\item Extra-curricular
\end{itemize}