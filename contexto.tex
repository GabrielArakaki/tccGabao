%!TEX root = index.tex
\chapter{Contexto}
\label{cha:contexto}

\section{Identificação dos Stakeholders}
\label{sec:identificacao_stakeholders}

Ao longo da implementação do Ocean, foram definidos os seguintes stakeholders:

\begin{itemize}
\item Samsung (Selber, Conejo)
\item PRO ( Chefia e COC [Zancul e Mesquita] e Pesquisa [Roberta])
\item Usuários Externos ( Reinaldo [IGDA], Sakuda [Abragames], IoT)
\item Alunos (CAEP, RD)
\end{itemize}

\section{Departamento do PRO}
\label{sec:dep_pro}

Segundo \citeonline{jeeModels}, o ensino de engenharia pode ser dividido em 3 principais modelos: Acadêmico, Market-Driven e Integrativo. <--------- INSERIR TABELA EXPLICATIVA ---------->

Embora exista uma tendência das universidades em trabalhar principalmente com o primeiro modelo, ele não deixa de ser um modelo idealizado, portanto cada vez mais é gerado espaço para os outros modelos através de iniciativas da gestão do ensino. Aulas de Empreendedorismo, Marketing e Desenvolvimento de Produto representam muito bem um modelo Market-Driven, ao passo que aulas de Sustentabilidade e Desenvolvimento de Problemas da sociedade representam bem o modelo Integrativo.

No caso do modelo \textit{market-driven}, são abertas as portas para estruturas mais práticas de ensino, e dentro desse contexto observa-se um papel fundamental do Ocean para o departamento: auxiliar no desenvolvimento de futuros engenheiros para estarem alinhados com as necessidades do mercado. 