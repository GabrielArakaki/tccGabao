%!TEX root = index.tex
\chapter[Revisão Bibliográfica]{Revisão Bibliográfica}
\label{chap:revisao}
\section{Estudos sobre ensino de engenharia}
\label{cha:ensino}

Segundo as palavras do professor diretor da Escola Politécnica: “A engenharia deve erradicar a pobreza gerando riqueza, através da geração de empregos e criação de empresas”. Tais palavras foram utilizadas na inauguração do laboratório Ocean em parceria entre o departamento de Engenharia de Produção (PRO) e a grande multinacional Samsung, tendo suas operações dentro da Escola Politécnica. O laboratório é uma parceria de cogestão entre universidade e empresa que tem como principal mérito a geração de valor derivada da sinergia entre as pesquisas da academia e o conhecimento aplicado da indústria. 

O Ocean é um dos quatro grandes projetos que o PRO acompanha atualmente, esquematizado pela figura a seguir: (DESENHAR FIGURA QUE SERA COLOCADA AQUI)

Inovalab, que existe e funciona, e é interno ao PRO

Fábrica Didática, que não existe mas seria interno ao PRo

Ocean, que existe e é de cogestão PRO Samsung

NEU, que existe e funciona mas é externo ao PRO

Portanto, é de grande interesse do PRO garantir que o laboratório esteja sendo utilizado da melhor forma possível, portanto é necessário mapear todos os interessados pelo bom funcionamento e traçar um plano de ações e indicadores de forma a garantir o pleno potencial do laboratório.
